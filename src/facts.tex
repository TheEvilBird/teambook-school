\begin{itemize}
    \item Собственное вращение на угол $\varphi$ с центром вращения в начале координат: \\ $x' = x \cos \varphi - y \sin \varphi \\ 
    y' = x \sin \varphi + y \cos \varphi$
    \item Расстояние между точками по сфере: 
    $L = R \cdot \arccos(\cos \theta_{1} \cdot \cos \theta_{2} + \sin \theta_{1} \cdot \sin \theta_{2} \cdot \cos(\varphi_{1} - \varphi_{2}))$
    где $\theta$ – широты (от $-\pi$ до $\pi$), $\varphi$ – долготы (от $-\pi$ до $\pi$)
    \item Объем шарового сегмента: $V = \pi h^{2}(R - \frac{1}{3}h)$, где $h$ -- высота от вершины сектора до секущей плоскости
    \item Площадь поверхности шарового сегмента: $S = 2\pi Rh$, где $h$ -- высота
    % \item $2^{23} \cdot 7 \cdot 17 + 1 = 998.244.353$ --- простое, первообразный корень -- $3$
    \item Код Грея: $g_{n} = n \oplus \frac{n}{2}$
    \item Числа Фибоначчи: \\ $\displaystyle F_{0} = 0, F_{1} = 1, F_{n} = \frac{(\frac{1 + \sqrt{5}}{2})^{n} - (\frac{1 - \sqrt{5}}{2})^{n}}{\sqrt{5}}$
    \item Sum-xor property: $a + b = a \oplus b + 2(a\&b), a + b = a|b + a\&b, a \oplus b = a|b - a\&b$
    \item Число граней в планарном графе(с учётом бесконечной): $R = 2 - V + E$
    \item Сумма арифметической прогрессии: $S_{n} = \frac{n(a_{1} + a_{n})}{2}$
    \item Сумма геометрической прогрессии: $S_{n} = \frac{b_{1}(q^{n} - 1)}{q - 1}$
    \item Определители матриц
    $$
    \left|
    \begin{array}{c c}
        a & b \\
        c & d
    \end{array}
    \right| = ad - bc
    $$
    \begin{multline*}
    \left|
    \begin{array}{c c c}
        a_{1} & b_{1} & c_{1} \\
        a_{2} & b_{2} & c_{2} \\
        a_{3} & b_{3} & c_{3} \\
    \end{array}
    \right| = a_{1}b_{1}c_{1} + a_{3}b_{1}c_{2} + a_{2}b_{3}c_{1} -\\- a_{3}b_{2}c_{1} - a_{1}b_{3}c_{2} - a_{2}b_{1}c_{3}
    \end{multline*}
    $\Delta = \sum_{j = 1}^{n} (-1)^{j + 1} \cdot a_{1, j} \cdot \bar{M_{j}^{1}}$, $\bar{M_{j}^{1}}$ --- определитель матрицы, полученной вычеркиванием $1$ строки и $j$ стоблца.
    \item \textbf{Метод Крамера.} $\det A \neq 0 \implies$ единственное решение. Иначе $0$ или $\infty$.
    Решения: $\displaystyle x_{i} = \frac{\Delta_{i}}{\Delta}$. В $\Delta_{i}$ столбец коэффициентов при соответствующей неизвестной заменяется столбцом свободных членов системы.
\end{itemize}
